\let\negmedspace\undefined
\let\negthickspace\undefined
\documentclass[journal,12pt,onecolumn]{IEEEtran}
\usepackage{cite}
\usepackage{amsmath,amssymb,amsfonts,amsthm}
\usepackage{algorithmic}
\usepackage{graphicx}
\usepackage{textcomp}
\usepackage{xcolor}
\usepackage{txfonts}
\usepackage{listings}
\usepackage{enumitem}
\usepackage{mathtools}
\usepackage{gensymb}
\usepackage{comment}
\usepackage{caption}
\usepackage[breaklinks=true]{hyperref}
\usepackage{tkz-euclide} 
\usepackage{listings}

\usepackage{gvv}                                        
%\def\inputGnumericTable{}                                 
\usepackage[latin1]{inputenc}     
\usepackage{xparse}
\usepackage{color}                                            
\usepackage{array}                                            
\usepackage{longtable}                                       
\usepackage{calc}                                             
\usepackage{multirow}
\usepackage{multicol}
\usepackage{hhline}                                           
\usepackage{ifthen}                                           
\usepackage{lscape}
\usepackage{tabularx}
\usepackage{array}
\usepackage{float}
%\newtheorem{theorem}{Theorem}[section]
%\newtheorem{theorem}{Theorem}[section]
%\newtheorem{problem}{Problem}
%\newtheorem{proposition}{Proposition}[section]
%\newtheorem{lemma}{Lemma}[section]
%\newtheorem{corollary}[theorem]{Corollary}
%\newtheorem{example}{Example}[section]
%\newtheorem{definition}[problem]{Definition}

\begin{document}

\title{12.529}
\author{AI25BTECH11035 - SUJAL RAJANI}
% \maketitle
% \newpage
% \bigskip
%\begin{document}
{\let\newpage\relax\maketitle}
%\renewcommand{\thefigure}{\theenumi}
%\renewcommand{\thetable}{\theenumi}
% \newpage
% \bigskip



\textbf{Question:}

Let $\mathbf{M}$ be a $3 \times 3$ matrix and suppose that $1, 2$ and $3$ are the eigenvalues of $\mathbf{M}$. If
\[
\mathbf{M}^{-1} = \frac{\mathbf{M}^2 - \mathbf{M} + \mathbf{I}_3}{\alpha}
\]
for some scalar $\alpha \neq 0$, then $\alpha$ is equal to \underline{\qquad}.

\vspace{1cm}

\textbf{Solution:}

Let $\lambda$ be an eigenvalue of $\mathbf{M}$, and $\mathbf{v}$ the corresponding eigenvector: 
\[
\mathbf{M} \mathbf{v} = \lambda \mathbf{v}
\]
Applying both sides to the given identity:
\[
\mathbf{M}^{-1} \mathbf{v} = \frac{\mathbf{M}^2 - \mathbf{M} + \mathbf{I}_3}{\alpha} \mathbf{v}
\]
Since polynomial expressions in $\mathbf{M}$ act on eigenvectors by substituting $\lambda$ for $\mathbf{M}$, we have:
\[
\mathbf{M}^2 \mathbf{v} = \lambda^2 \mathbf{v}, \qquad \mathbf{I}_3 \mathbf{v} = \mathbf{v}
\]
Thus,
\[
\mathbf{M}^{-1} \mathbf{v} = \frac{\lambda^2 - \lambda + 1}{\alpha} \mathbf{v}
\]
But 
\[
\mathbf{M}^{-1} \mathbf{v} = \frac{1}{\lambda} \mathbf{v}
\]
Therefore,
\[
\frac{1}{\lambda} = \frac{\lambda^2 - \lambda + 1}{\alpha}
\implies
\alpha = \lambda (\lambda^2 - \lambda + 1)
\]
For the three eigenvalues, compute $\alpha$:

For $\lambda = 1$:
\[
\alpha = 1 \times (1^2 - 1 + 1) = 1
\]
For $\lambda = 2$:
\[
\alpha = 2 \times (2^2 - 2 + 1) = 2 \times (4 - 2 + 1) = 6
\]
For $\lambda = 3$:
\[
\alpha = 3 \times (9 - 3 + 1) = 3 \times 7 = 21
\]
Thus, there is no single $\alpha$ that satisfies all three cases, so the expression does not hold for all eigenvalues simultaneously.

\end{document}

\end{document}

\end{document}
