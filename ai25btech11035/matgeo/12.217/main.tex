\let\negmedspace\undefined
\let\negthickspace\undefined
\documentclass[journal,12pt,onecolumn]{IEEEtran}
\usepackage{cite}
\usepackage{amsmath,amssymb,amsfonts,amsthm}
\usepackage{algorithmic}
\usepackage{graphicx}
\usepackage{textcomp}
\usepackage{xcolor}
\usepackage{txfonts}
\usepackage{listings}
\usepackage{enumitem}
\usepackage{mathtools}
\usepackage{gensymb}
\usepackage{comment}
\usepackage{caption}
\usepackage[breaklinks=true]{hyperref}
\usepackage{tkz-euclide} 
\usepackage{listings}

\usepackage{gvv}                                        
%\def\inputGnumericTable{}                                 
\usepackage[latin1]{inputenc}     
\usepackage{xparse}
\usepackage{color}                                            
\usepackage{array}                                            
\usepackage{longtable}                                       
\usepackage{calc}                                             
\usepackage{multirow}
\usepackage{multicol}
\usepackage{hhline}                                           
\usepackage{ifthen}                                           
\usepackage{lscape}
\usepackage{tabularx}
\usepackage{array}
\usepackage{float}
%\newtheorem{theorem}{Theorem}[section]
%\newtheorem{theorem}{Theorem}[section]
%\newtheorem{problem}{Problem}
%\newtheorem{proposition}{Proposition}[section]
%\newtheorem{lemma}{Lemma}[section]
%\newtheorem{corollary}[theorem]{Corollary}
%\newtheorem{example}{Example}[section]
%\newtheorem{definition}[problem]{Definition}

\begin{document}

\title{12.217}
\author{AI25BTECH11035 - SUJAL RAJANI}
% \maketitle
% \newpage
% \bigskip
%\begin{document}
{\let\newpage\relax\maketitle}
%\renewcommand{\thefigure}{\theenumi}
%\renewcommand{\thetable}{\theenumi}
% \newpage
% \bigskip
\section*{Directional Derivative at a Point on a Surface}

Given the surface
\[
x^2 + \frac{y^2}{4} + \frac{z^2}{9} = 3,
\]
find the directional derivative at \(P(1,2,3)\) in the direction of the vector $\vec{P}$, where \(O\) is the origin.

\subsection*{Step 1: Theorem for Directional Derivative}
\\
we represent the partial derivative as :
\\
\begin{align*}
    \frac{\partial f}{\partial x}=f_x(x,y)
\end{align*} 
\\
 the directional derivative of \(f(x, y)\) in the direction of the unit vector $\vec{u}$ =  \myvec{a\\ b} is
\[
D_{\mathbf{u}} f(x, y) = f_x(x, y)a + f_y(x, y)b,
\]
and in three dimensions,
\[
D_{\mathbf{u}} F(x, y, z) = \vec{\nabla F(x, y, z)}^\top  \vec{u},
\]
where $\vec{\nabla F}$ is the gradient of \(F\).

\subsection*{Step 2: Find the Gradient}

Let
\[
F(x, y, z) = x^2 + \frac{y^2}{4} + \frac{z^2}{9}.
\]
Compute the gradient:
\[
\vec{\nabla F} = \left( \frac{\partial F}{\partial x}, \frac{\partial F}{\partial y}, \frac{\partial F}{\partial z} \right)^\top
= \myvec{2x& \frac{y}{2}& \frac{2z}{9}}^\top
\]

At \(P(1,2,3)\):
\begin{align*}
\vec{\nabla F(1,2,3)} =  \myvec{2&1&\frac{2}{3}}^\top
\end{align*}

\textbf{Step 3: Unit Direction Vector}
\\
The position vector from the origin to the point \(P(1,2,3)\) is
\begin{align*}
    \vec{P}= \myvec{1\\2\\3}
\end{align*}
To get the unit vector,
\begin{align*}
||\vec{P}||^2 =\vec{P}^\top\vec{P}=1^2 + 2^2 + 3^2 = {14} 
\\
\hat{P}=\dfrac{\vec{P}}{||\vec{P}||}
\end{align*}

So,
\[
\vec{P} = \left(\frac{1}{\sqrt{14}}, \frac{2}{\sqrt{14}}, \frac{3}{\sqrt{14}}\right)^\top
\]

\subsection*{Step 4: Directional Derivative Formula and Calculation}

By the theorem :
\[
D_{\mathbf{u}} F = \vec{{\nabla F}}^\top  \vec{P}
\]
\[
=  \frac{2}{\sqrt{14}} + \frac{2}{\sqrt{14}} +  \frac{2}{\sqrt{14}}
\]
\[
= \frac{6}{\sqrt{14}}
\]

\subsection*{Final Answer}
\[
\boxed{ \frac{6}{\sqrt{14}} }
\]

\end{document}
