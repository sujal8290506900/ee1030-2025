\documentclass{beamer}
\usepackage[utf8]{inputenc}

\usetheme{Madrid}
\usecolortheme{default}
\usepackage{amsmath,amssymb,amsfonts,amsthm}
\usepackage{mathtools}
\usepackage{txfonts}
\usepackage{tkz-euclide}
\usepackage{listings}
\usepackage{adjustbox}
\usepackage{array}
\usepackage{gensymb}
\usepackage{tabularx}
\usepackage{gvv}
\usepackage{lmodern}
\usepackage{circuitikz}
\usepackage{tikz}
\lstset{literate={·}{{$\cdot$}}1 {λ}{{$\lambda$}}1 {→}{{$\to$}}1}
\usepackage{graphicx}

\setbeamertemplate{page number in head/foot}[totalframenumber]

\usepackage{tcolorbox}
\tcbuselibrary{minted,breakable,xparse,skins}



\definecolor{bg}{gray}{0.95}
\DeclareTCBListing{mintedbox}{O{}m!O{}}{%
  breakable=true,
  listing engine=minted,
  listing only,
  minted language=#2,
  minted style=default,
  minted options={%
    linenos,
    gobble=0,
    breaklines=true,
    breakafter=,,
    fontsize=\small,
    numbersep=8pt,
    #1},
  boxsep=0pt,
  left skip=0pt,
  right skip=0pt,
  left=25pt,
  right=0pt,
  top=3pt,
  bottom=3pt,
  arc=5pt,
  leftrule=0pt,
  rightrule=0pt,
  bottomrule=2pt,
  toprule=2pt,
  colback=bg,
  colframe=orange!70,
  enhanced,
  overlay={%
    \begin{tcbclipinterior}
    \fill[orange!20!white] (frame.south west) rectangle ([xshift=20pt]frame.north west);
    \end{tcbclipinterior}},
  #3,
}
\lstset{
    language=C,
    basicstyle=\ttfamily\small,
    keywordstyle=\color{blue},
    stringstyle=\color{orange},
    commentstyle=\color{green!60!black},
    numbers=left,
    numberstyle=\tiny\color{gray},
    breaklines=true,
    showstringspaces=false,
}
%------------------------------------------------------------
%This block of code defines the information to appear in the
%Title page
\title %optional
{5.13.21}
\date{September 5,2025}
%\subtitle{A short story}

\author % (optional)
{AI25BTECH11035-SUJAL RAJANI}
\begin{document}
\frame{\titlepage}
\begin{frame}{Question}
if 
\begin{align*}
    \myvec{1&1\\0&1}.\myvec{1&2\\0&1}.\myvec{1&3\\0&1}\cdots\myvec{1&n-1\\0&1}=\myvec{1&78\\0&1}.
\end{align*}
\\
then the inverse of $\myvec{1&n\\0&1}$ is 
  \\
  (a) $\myvec{1&0\\12&1}$
  \\
  \\
  (b) $\myvec{1&-13\\0&1}$
  \\
  \\
  (c) $\myvec{1&-12\\0&1}$
  \\
  \\
  (d) $\myvec{1&0\\13&1}$
  \\
\end{frame}

\begin{frame}{Theoretical Solution}
as mentioned in the question :
\\
\begin{align*}
    \myvec{1&1\\0&1}.\myvec{1&2\\0&1}.\myvec{1&3\\0&1}.\myvec{1&n-1\\0&1}=\myvec{1&78\\0&1}.
\end{align*}
let:
\begin{equation*}
   \Vec{A}=\myvec{1&n\\0&1}
\end{equation*}
in these type of question we start the solution by analyzing the pattern. 
\end{frame}
\begin{frame}{Theoretical solution}
\begin{align*}
    \myvec{1&1\\0&1}.\myvec{1&2\\0&1}=\myvec{1&3\\0&1}
\end{align*}
\begin{align*}
   \myvec{1&3\\0&1}.\myvec{1&3\\0&1}=\myvec{1&6\\0&1}
\end{align*}
\begin{align*}
   \myvec{1&6\\0&1}.\myvec{1&4\\0&1}=\myvec{1&10\\0&1}
\end{align*}
\begin{align*}
   \myvec{1&10\\0&1}.\myvec{1&5\\0&1}=\myvec{1&15\\0&1}
\end{align*}
\end{frame}

\begin{frame}{Theoretical Solution}
let the resulting matrix is in the format :
\begin{align*}
    \myvec{\alpha_{11}&\alpha_{12}\\\alpha_{21}&\alpha_{22}}
    \\
    \alpha_{11}=1,\alpha_{12}=\dfrac{n(n+1)}{2},\alpha_{21}=0,\alpha_{22}=1
\end{align*}
so replacing n with n-1 :
\begin{align*}
    \myvec{1&\dfrac{n(n-1)}{2}\\0&1}=\myvec{1&78\\0&1}
    \\
    \dfrac{n(n-1)}{2}=78
    \end{align*}
    \begin{equation*}
        n=13
    \end{equation*}
\end{frame}
\begin{frame}{Inverse of a matrix}
 \textbf{inverse of a matrix}
    \begin{align*}
        \vec{Adj(A)}\vec{A}=||\vec{A}||\vec{I}
        \\
        \vec{Adj(A)}=||\vec{A}||\vec{A}^{-1}
        \\
        \vec{A}^{-1}=\dfrac{\vec{Adj(A)}}{||\vec{A}||}
    \end{align*}
\end{frame}
\begin{frame}{Theoretical Solution}
 adjoint of $\vec{A}$ matrix is :
    \begin{align*}
        \vec{Adj(A)}=\myvec{1&-n\\0&1}
        \\
        ||\Vec{A}||=1
        \\
        \vec{A}^{-1}=\myvec{1&-n\\0&1}
    \end{align*}
    \begin{align*}
        \vec{A}^{-1}=\myvec{1&-13\\0&1}
    \end{align*}
    option (B) is correct .
\end{frame}
\begin{frame}[fragile]
    \frametitle{C Code -Finding Solution for the system of Equations}

    \begin{lstlisting}[language=C]

#include <stdio.h>

int main() {
    int n;
    int sum = 0;

    // Find n such that (n-1)*n/2 = 78
    for (n = 1; n < 100; n++) {
        sum = (n - 1) * n / 2;
        if (sum == 78) {
            break;
        }
    }

    \end{lstlisting}
\end{frame}

\begin{frame}[fragile]
    \frametitle{C Code -Finding Solution for the system of Equations}

    \begin{lstlisting}[language=C]
    
 printf("Value of n = %d\n", n);

    // Inverse matrix of [[1, n], [0, 1]]
    int inv[2][2] = {{1, -n}, {0, 1}};

    printf("Inverse matrix is:\n");
    printf("[[%d, %d],\n [%d, %d]]\n",
           inv[0][0], inv[0][1], inv[1][0], inv[1][1]);

    return 0;
}


    \end{lstlisting}
\end{frame}





\end{document}