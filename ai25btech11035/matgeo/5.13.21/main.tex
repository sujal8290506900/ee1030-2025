\let\negmedspace\undefined
\let\negthickspace\undefined
\documentclass[journal,12pt,onecolumn]{IEEEtran}
\usepackage{cite}
\usepackage{amsmath,amssymb,amsfonts,amsthm}
\usepackage{algorithmic}
\usepackage{graphicx}
\usepackage{textcomp}
\usepackage{xcolor}
\usepackage{txfonts}
\usepackage{listings}
\usepackage{enumitem}
\usepackage{mathtools}
\usepackage{gensymb}
\usepackage{comment}
\usepackage{caption}
\usepackage[breaklinks=true]{hyperref}
\usepackage{tkz-euclide} 
\usepackage{listings}

\usepackage{gvv}                                        
%\def\inputGnumericTable{}                                 
\usepackage[latin1]{inputenc}     
\usepackage{xparse}
\usepackage{color}                                            
\usepackage{array}                                            
\usepackage{longtable}                                       
\usepackage{calc}                                             
\usepackage{multirow}
\usepackage{multicol}
\usepackage{hhline}                                           
\usepackage{ifthen}                                           
\usepackage{lscape}
\usepackage{tabularx}
\usepackage{array}
\usepackage{float}
%\newtheorem{theorem}{Theorem}[section]
%\newtheorem{theorem}{Theorem}[section]
%\newtheorem{problem}{Problem}
%\newtheorem{proposition}{Proposition}[section]
%\newtheorem{lemma}{Lemma}[section]
%\newtheorem{corollary}[theorem]{Corollary}
%\newtheorem{example}{Example}[section]
%\newtheorem{definition}[problem]{Definition}

\begin{document}

\title{5.13.21}
\author{AI25BTECH11035 - SUJAL RAJANI}
% \maketitle
% \newpage
% \bigskip
%\begin{document}
{\let\newpage\relax\maketitle}
%\renewcommand{\thefigure}{\theenumi}
%\renewcommand{\thetable}{\theenumi}
% \newpage
% \bigskip
\textbf{QUESTION}
\\
if 
\begin{align*}
    \myvec{1&1\\0&1}.\myvec{1&2\\0&1}.\myvec{1&3\\0&1}\cdots\myvec{1&n-1\\0&1}=\myvec{1&78\\0&1}.
\end{align*}
\\
then the inverse of $\myvec{1&n\\0&1}$ is 
  \\
  (a) $\myvec{1&0\\12&1}$
  \\
  \\
  (b) $\myvec{1&-13\\0&1}$
  \\
  \\
  (c) $\myvec{1&-12\\0&1}$
  \\
  \\
  (d) $\myvec{1&0\\13&1}$
  \\
\textbf{solution}
\\
as mentioned in the question :
\\
\begin{align*}
    \myvec{1&1\\0&1}.\myvec{1&2\\0&1}.\myvec{1&3\\0&1}.\myvec{1&n-1\\0&1}=\myvec{1&78\\0&1}.
\end{align*}
let:
\begin{equation*}
   \Vec{A}=\myvec{1&n\\0&1}
\end{equation*}
in these type of question we start the solution by analyzing the pattern. 
\\
\begin{align*}
    \myvec{1&1\\0&1}.\myvec{1&2\\0&1}=\myvec{1&3\\0&1}
\end{align*}
\begin{align*}
   \myvec{1&3\\0&1}.\myvec{1&3\\0&1}=\myvec{1&6\\0&1}
\end{align*}
\begin{align*}
   \myvec{1&6\\0&1}.\myvec{1&4\\0&1}=\myvec{1&10\\0&1}
\end{align*}
\begin{align*}
   \myvec{1&10\\0&1}.\myvec{1&5\\0&1}=\myvec{1&15\\0&1}
\end{align*}
let the resulting matrix is in the format :
\begin{align*}
    \myvec{\alpha_{11}&\alpha_{12}\\\alpha_{21}&\alpha_{22}}
    \\
    \alpha_{11}=1,\alpha_{12}=\dfrac{n(n+1)}{2},\alpha_{21}=0,\alpha_{22}=1
\end{align*}
so replacing n with n-1 :
\begin{align*}
    \myvec{1&\dfrac{n(n-1)}{2}\\0&1}=\myvec{1&78\\0&1}
    \\
    \dfrac{n(n-1)}{2}=78
    \end{align*}
    \begin{equation*}
        n=13
    \end{equation*}
    \textbf{inverse of a matrix}
    \begin{align*}
        \vec{Adj(A)}\vec{A}=||\vec{A}||\vec{I}
        \\
        \vec{Adj(A)}=||\vec{A}||\vec{A}^{-1}
        \\
        \vec{A}^{-1}=\dfrac{\vec{Adj(A)}}{||\vec{A}||}
    \end{align*}
    adjoint of $\vec{A}$ matrix is :
    \begin{align*}
        \vec{Adj(A)}=\myvec{1&-n\\0&1}
        \\
        ||\Vec{A}||=1
        \\
        \vec{A}^{-1}=\myvec{1&-n\\0&1}
    \end{align*}
    by equation (1) we are using the value of n :
    \begin{align*}
        \vec{A}^{-1}=\myvec{1&-13\\0&1}
    \end{align*}
    option (B) is correct .
   
  

\end{document}