\let\negmedspace\undefined
\let\negthickspace\undefined
\documentclass[journal,12pt,onecolumn]{IEEEtran}
\usepackage{cite}
\usepackage{amsmath,amssymb,amsfonts,amsthm}
\usepackage{algorithmic}
\usepackage{graphicx}
\usepackage{textcomp}
\usepackage{xcolor}
\usepackage{txfonts}
\usepackage{listings}
\usepackage{enumitem}
\usepackage{mathtools}
\usepackage{gensymb}
\usepackage{comment}
\usepackage{caption}
\usepackage[breaklinks=true]{hyperref}
\usepackage{tkz-euclide} 
\usepackage{listings}

\usepackage{gvv}                                        
%\def\inputGnumericTable{}                                 
\usepackage[latin1]{inputenc}     
\usepackage{xparse}
\usepackage{color}                                            
\usepackage{array}                                            
\usepackage{longtable}                                       
\usepackage{calc}                                             
\usepackage{multirow}
\usepackage{multicol}
\usepackage{hhline}                                           
\usepackage{ifthen}                                           
\usepackage{lscape}
\usepackage{tabularx}
\usepackage{array}
\usepackage{float}
%\newtheorem{theorem}{Theorem}[section]
%\newtheorem{theorem}{Theorem}[section]
%\newtheorem{problem}{Problem}
%\newtheorem{proposition}{Proposition}[section]
%\newtheorem{lemma}{Lemma}[section]
%\newtheorem{corollary}[theorem]{Corollary}
%\newtheorem{example}{Example}[section]
%\newtheorem{definition}[problem]{Definition}

\begin{document}

\title{5.7.8}
\author{AI25BTECH11035 - SUJAL RAJANI}
% \maketitle
% \newpage
% \bigskip
%\begin{document}
{\let\newpage\relax\maketitle}
%\renewcommand{\thefigure}{\theenumi}
%\renewcommand{\thetable}{\theenumi}
% \newpage
% \bigskip
\textbf{QUESTION}
\\
  Given that $\vec{A}$ =$\myvec{\alpha&\beta\\\gamma&-\alpha}$ and $\vec{A}^2$=3$\vec{I}$,then
  \\
  (a) $1+\alpha^2+\beta\gamma$ =0
  \\
  (b) $1-\alpha^2-\beta\gamma$ =0
  \\
  (c) $3-\alpha^2-\beta\gamma$ =0
  \\
  (d) $3+\alpha^2+\beta\gamma$ =0
  \\
\textbf{solution}
as mentioned in the question :
\\
\begin{align*}
   \vec{A} =\myvec{\alpha&\beta\\\gamma&-\alpha}
   \end{align*}
\begin{equation}
   \vec{A}^2 &=3\vec{I}.
\end{equation}
\\
The characteristic equation of $\vec{A}$ is:
\begin{align}
\left|\myvec{
\alpha-\lambda & \beta  \\
\gamma & -\alpha-\lambda  
}\right| &=0 ,
\end{align}
which expands to
\begin{align}
\lambda^2 -  \alpha^2 -\beta\gamma= 0.
\end{align}
 By Carlyle-Hamilton theorem
    \\
    \begin{align*}
    \vec{A}^2=(\alpha^2 +\beta\gamma)\vec{I}
    \end{align*}
    by equation 1 :
    \\
    \begin{align}
     3\vec{I}=(\alpha^2 +\beta\gamma)\vec{I}
     \end{align}
     \begin{align*}
     3=\alpha^2 + \beta\gamma
      \end{align*}.
     option c is the correct option .
     
\end{document}