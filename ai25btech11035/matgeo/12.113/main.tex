\let\negmedspace\undefined
\let\negthickspace\undefined
\documentclass[journal,12pt,onecolumn]{IEEEtran}
\usepackage{cite}
\usepackage{amsmath,amssymb,amsfonts,amsthm}
\usepackage{algorithmic}
\usepackage{graphicx}
\usepackage{textcomp}
\usepackage{xcolor}
\usepackage{txfonts}
\usepackage{listings}
\usepackage{enumitem}
\usepackage{mathtools}
\usepackage{gensymb}
\usepackage{comment}
\usepackage{caption}
\usepackage[breaklinks=true]{hyperref}
\usepackage{tkz-euclide} 
\usepackage{listings}

\usepackage{gvv}                                        
%\def\inputGnumericTable{}                                 
\usepackage[latin1]{inputenc}     
\usepackage{xparse}
\usepackage{color}                                            
\usepackage{array}                                            
\usepackage{longtable}                                       
\usepackage{calc}                                             
\usepackage{multirow}
\usepackage{multicol}
\usepackage{hhline}                                           
\usepackage{ifthen}                                           
\usepackage{lscape}
\usepackage{tabularx}
\usepackage{array}
\usepackage{float}
%\newtheorem{theorem}{Theorem}[section]
%\newtheorem{theorem}{Theorem}[section]
%\newtheorem{problem}{Problem}
%\newtheorem{proposition}{Proposition}[section]
%\newtheorem{lemma}{Lemma}[section]
%\newtheorem{corollary}[theorem]{Corollary}
%\newtheorem{example}{Example}[section]
%\newtheorem{definition}[problem]{Definition}

\begin{document}

\title{12.113}
\author{AI25BTECH11035 - SUJAL RAJANI}
% \maketitle
% \newpage
% \bigskip
%\begin{document}
{\let\newpage\relax\maketitle}
%\renewcommand{\thefigure}{\theenumi}
%\renewcommand{\thetable}{\theenumi}
% \newpage
% \bigskip


\textbf{Problem:} \\
The area of the region bounded by the parabola \( y = x^2 + 1 \) and the straight line \( x + y = 3 \) is:

\begin{enumerate}
    \item[\textbf{a)}] \( \frac{59}{6} \) 
    \item[\textbf{b)}] \( \frac{9}{2} \)
    \item[\textbf{c)}] \( \frac{10}{3} \)
    \item[\textbf{d)}] \( \frac{7}{6} \)
\end{enumerate}

\textbf{Solution:}

\textbf{Step 1: Write equations in matrix (quadratic) form}
\begin{align*}
y &= x^2 + 1 \implies x^2 - y + 1 = 0 \\
\text{In matrix form:} \quad \vec{x}^T 
\begin{pmatrix} 1 & 0 \\ 0 & 0 \end{pmatrix}
\vec{x} + 2
\begin{pmatrix} 0 \\ -\frac{1}{2} \end{pmatrix}^T
\vec{x} + 1 = 0
\end{align*}

\textbf{Step 2: Parametric representation of the line:} \\
\( x + y = 3 \implies y = 3 - x \) \\
Let the line in parametric vector form be:
\[
\vec{X} = \begin{pmatrix} 0 \\ 3 \end{pmatrix} + \lambda \begin{pmatrix} 1 \\ -1 \end{pmatrix}
\]
So,
\[
x = \lambda,\quad y = 3 - \lambda
\]

\textbf{Step 3: Substitute into the matrix equation and solve for \(\lambda\)}

Substitute \( x = \lambda,\, y = 3 - \lambda \) into the matrix equation.
\begin{align*}
\left(
\begin{pmatrix}
\lambda \\
3 - \lambda
\end{pmatrix}
\right)^T 
\begin{pmatrix}
1 & 0 \\
0 & 0
\end{pmatrix}
\begin{pmatrix}
\lambda \\
3 - \lambda
\end{pmatrix}
+ 2
\begin{pmatrix}
0 \\
-\frac{1}{2}
\end{pmatrix}^T 
\begin{pmatrix}
\lambda \\
3 - \lambda
\end{pmatrix}
+ 1 = 0
\end{align*}
Calculate each component:
\begin{align*}
& \begin{pmatrix}
\lambda \\
3 - \lambda
\end{pmatrix}^T 
\begin{pmatrix}
1 & 0 \\
0 & 0
\end{pmatrix}
\begin{pmatrix}
\lambda \\
3 - \lambda
\end{pmatrix}
= \lambda^2 \\
& 2\begin{pmatrix}
0 \\
-\frac{1}{2}
\end{pmatrix}^T 
\begin{pmatrix}
\lambda \\
3 - \lambda
\end{pmatrix}
= 2 \cdot \left(0 \cdot \lambda + \left(-\frac{1}{2}\right) (3 - \lambda) \right)
= - (3 - \lambda) = -3 + \lambda
\end{align*}
So, the equation becomes:
\[
\lambda^2 - 3 + \lambda + 1 = 0 
\implies \lambda^2 + \lambda - 2 = 0
\]
Solving this quadratic equation:
\[
(\lambda + 2)(\lambda - 1) = 0 \implies \lambda = -2,\, 1
\]
So, the intersection points are:
\[
\vec{X} = \myvec{-2\\ 5} , \vec{X} = \myvec{1\\2}
\]

\textbf{Step 4: Area Calculation}

The area between the curves can be written as:
\[
A = \int_{-2}^{1} \left[ (3-x) - (x^2 + 1) \right] \, dx = \int_{-2}^{1} (2 - x - x^2)\, dx
\]

Integrate:
\begin{align*}
A &= \int_{-2}^{1} (2 - x - x^2)\,dx \\
&= \left[ 2x - \frac{1}{2}x^2 - \frac{1}{3}x^3 \right]_{-2}^{1}\\
&= \left(2(1) - \frac{1}{2}(1)^2 - \frac{1}{3}(1)^3\right) - \left(2(-2) - \frac{1}{2}(-2)^2 - \frac{1}{3}(-2)^3\right) \\
&= (2 - 0.5 - 0.333) - (-4 - 2 + 2.666) \\
&= 1.167 - (-3.334) \\
&= 4.5 = \frac{9}{2}
\end{align*}

\textbf{Final Answer:}
\[
\boxed{\frac{9}{2}}
\]

\end{document}

\end{document}
